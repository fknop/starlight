\documentclass[]{article}
\usepackage{amssymb,amsmath}

\usepackage[utf8]{inputenc}
\usepackage[francais]{babel}

\usepackage[unicode=true,
	colorlinks=true,
linkcolor=blue]{hyperref}

\hypersetup{breaklinks=true, pdfborder={0 0 0}}
\setlength{\parindent}{0pt}
\setlength{\parskip}{6pt plus 2pt minus 1pt}
\setlength{\emergencystretch}{3em}  % prevent overfull lines
\setcounter{secnumdepth}{0}

\title{Rapport du projet Starlight}
\author{Florian Knop (39310) - Gatien Bovyn (39189)}

\begin{document}
\maketitle

\newpage

\tableofcontents

\newpage

\section{Introduction}


Ce document vise à présenter le travail d’analyse et de programmation effectué
lors de la réalisation du projet du laboratoire Langage C++ : Starlight.

Ce projet a été réalisé en binôme par Florian Knop, matricule 39310 groupe 2G13,
et Gatien Bovyn, matricule 39189 groupe 2G11.

Le programme à concevoir consiste en une implémentation du modèle et d’une interface
graphique du jeu baptisé Starlight, puzzle à 2 dimensions basé sur la lumière.

Ce projet a été compilé principalement avec g++
(version 4.8.2 ou supérieure)
sous la distribution Linux Ubuntu (ou une de ses dérivée).
La version du framework Qt utilisée est la 5.0.2 ou supérieure.
Ce projet a été fait sous QtCreator, IDE OpenSource en version 2.8.1 ou
supérieure.

\section{Sections}

\section{Conventions de nommages utilisées}

Dans cette section, nous présenterons les différentes conventions 
utilisées lors de ce projet. Nous avons décidé de reprendre certaines
conventions utilisées par la STL (Librairie Standard)
tout en utilisant d'autres conventions.

De manière globale, tous les noms de variables, classes, fichiers, etc.
sont en anglais.

\subsection{Nommage des fichiers}

Les noms de fichiers sont entièrement en minuscules
et possèdent les extensions .h
pour les headers et .cpp pour les fichiers sources.

\begin{itemize}
	\item \texttt{nomfichier.h}
	\item \texttt{nomfichier.cpp}
\end{itemize}

\subsection{Classes}

Les noms des classes commencent par une majuscule à chaque mot.
Cette convention est également valable pour les noms 
d'énumérations et structures.

\begin{itemize}
	\item \texttt{Classe}
	\item \texttt{NomClasse}
	\item \texttt{NomStruct}
	\item \texttt{NomEnumeration}
\end{itemize}

\newpage

\subsection{Variables}

\subsubsection{Variables de classe}

Les noms des variables de classes sont en minuscules, les mots sont 
séparés par un underscore et le nom est suffixé par un underscore.

\begin{itemize}
	\item \texttt{variable\_}
	\item \texttt{nom\_variable\_}
\end{itemize}

Cette convention est également valable pour les variables d'une
structure.

\subsubsection{Variables locales}

Les noms des variables locales respectent les mêmes conventions 
que les variables de classe, à la seule exception qu’ils ne sont pas
suffixés par un underscore.

\begin{itemize}
	\item \texttt{variable}
	\item \texttt{nom\_variable}
\end{itemize}

\subsubsection{Constantes}

Les noms des constantes sont entièrement en majuscules et les mots sont
séparés par des underscores.

\begin{itemize}
	\item \texttt{CONSTANTE}
	\item \texttt{NOM\_CONSTANTE}
\end{itemize}

\subsection{Méthodes}

D'une manière générale, les noms de méthodes sont en minuscules et
séparés par des underscores.

\subsubsection{Getters}

Le nom d'un getter (accesseur en lecture de variable de classe/structure)
est égal au nom de la variable de classe sans l'underscore final.

\begin{itemize}
	\item Variable : \texttt{variable\_} - Getter : \texttt{variable()}
	\item Variable : \texttt{nom\_var\_} - Getter : \texttt{nom\_var()}
\end{itemize}

\subsubsection{Setters}

Le nom d'un setter (accesseur en écriture de variable de classe/structure)
est égal au nom de la variable sans l'underscore final préfixé de \texttt{set}.

\begin{itemize}
	\item Variable : \texttt{variable\_} - Setter : \texttt{set\_variable()}
	\item Variable : \texttt{nom\_var\_} - Setter : \texttt{set\_nom\_var()}
\end{itemize}

\subsubsection{Autres méthodes}

Les autres méthodes possèdent les mêmes conventions que celles
énoncées ci-dessus.

\begin{itemize}
	\item \texttt{methode()}
	\item \texttt{nom\_methode()}
\end{itemize}

\subsection{Éléments d'une énumération} 

Les éléments d'une énumération suivent les mêmes conventions que celles des constantes énoncées ci-dessus.

\newpage
\section{Présentation générale du projet}

Cette section décrit les objectifs principaux et secondaires effectués lors
de la réalisation de ce projet.
Toutes les classes utilisées sont décrites dans la section suivante.

Le projet fini ouvre sur un menu principal permettant
de jouer à Starlight, de voir les rêgles du jeu, d'accéder à l'éditeur de 
carte ou de simplement quitter.

\subsection{Le jeu}

Le projet permet de jouer à Starlight en important sa propre carte de jeu
au format \texttt{.lvl} qu'il est possible de créer soi-même grâce à l'éditeur
de carte (cf. Editeur de cartes).

Il est nécessaire de préciser que le jeu part du principe que la carte 
fournie est sans erreurs. Une carte fournie avec erreur produira donc
un arrêt immédiat de l'application.

\subsubsection{Le fonctionnement du jeu}

Le but du jeu est de déplacer un rayon provenant d'une source lumineuse
pouvant être allumée ou éteinte
vers une destination à l'aide de miroirs plans amovibles réfléchissant
la lumière.
La lumière possède une longueur d'onde comprise dans le spectre
de lumière visible.

En addition avec leur déplacement, les miroirs peuvent tourner
autour d'un point de pivot.
Ce déplacement et cette rotation se font tout en restant
dans certaines limites si celles-ci ont été définies à la création du miroir.

Une gestion des collisions a également été implémentée pour
éviter que les miroirs déplacés ne heurtent les autres éléments de la carte.
De ce fait, la carte de jeu fournie par Mr. Absil a légèrement été modifiée
pour éviter que les miroirs ne se trouvent dans les murs à la création du niveau
et qu'il soit donc impossible de les déplacer par la suite.

La carte peut aussi posséder des cristaux qui modifient la longueur d'onde
du rayon les traversant, ainsi que des lentilles laissant passer la lumière
d'une certaine intervalle de longueur d'onde.
La couleur de la lumière est donc modifiée selon la longueur d'onde 
du rayon lumineux.

Et pour finir, la carte peut être munie de bombes qui terminent
instantanément la partie si celles-ci sont touchées. 


\subsubsection{L'interface}

L'interface graphique du jeu Starlight a été réalisée sous Qt (en version 5.0.2 et supérieures).
Il s'agit ici d'une interface simple et minimaliste permettant d'effetuer la fonction de base 
demandée : jouer.

La fenêtre de jeu possède également des menus permettants de :

\begin{itemize}
    \item \texttt{Quitter}
    \item \texttt{Revenir au menu principal}
    \item \texttt{Charger une carte}
    \item \texttt{Quitter la carte}
    \item \texttt{Voir les règles de jeu}
\end{itemize}

Les raccourcis clavier permettant d'accéder au fonction citées ci-dessus sont indépendants
du système d'exploitation utilisé.

Les miroirs sont sont sélectionnés en utilisant un double clic gauche.
Si un troisième clic suit, les limites de déplacement du pivot s'affichent
sous la forme d'un rectangle bleu.

Les miroirs peuvent être déplacés et tournés au clavier en utilisant les touches :

\begin{itemize}
	\item \texttt{Z} pour déplacer vers le haut
	\item \texttt{S} pour déplacer vers le bas
	\item \texttt{Q} pour déplacer vers la gauche
	\item \texttt{D} pour déplacer vers la droite
	\item \texttt{Flèche directionnelle gauche} pour tourner dans le sens
		anti-horloger
	\item \texttt{Flèche directionnelle droite} pour tourner dans le sens
		horloger
    \item \texttt{SHIFT} + une des touches citées ci-dessus pour se déplacer / tourner
        plus vite
\end{itemize}

Des sons ont été ajoutés lorsque la source est allumée ou éteinte, 
lorsqu'une bombe a été touchée et lorsque la destination
est atteinte (cf. Faiblesses de l'interface pour plus d'informations).

\paragraph{Faiblesses de l'interface} 

L'interface ne permet pas de déplacer les miroirs à la souris, mais seulement au clavier.
Le rectangle bleu affichant les limites du miroir s'affiche après un 3ème clic plutôt 
que lors de la sélection du miroir.

Les sons ajoutés qui ont été cités ci-dessus ne s'activent toujours.
Par exemple lorsque la source est allumée et éteinte très vite, 
mais il se peut également que la source n'émette simplement pas de son.


\subsection{Editeur de cartes}

L’éditeur de carte a été conçu sur base du modèle et des vues existantes.
Il reprend tous les éléments disponibles dans un niveau du jeu Starlight.

\subsubsection{Possibilités de l’éditeur}

Au lancement de l’éditeur, il est possible de charger un niveau existant 
ou d’en créer un nouveau en personnalisant sa taille. Celle-ci ne peut 
être changée par la suite.

Une fois chargé/créé, il est alors possible d’ajouter de nouveaux éléments 
dans le niveau (mur, miroir, lentille, cristal, bombe). Comme prévu par l’énoncé,
une seule source et une seule destination peuvent être incluses. Dès lors, 
celles-ci sont créées en même temps que le niveau et il n’est pas possible de 
les supprimer. Il n'est également pas possible de supprimer les murs extérieurs
au niveau.

Tous les éléments ont la possibilité d’être modifiés à tous niveaux, les setters
appropriés ayant été ajoutés dans les classes du modèle. Il est au minimum possible de déplacer
tous les éléments, soit par le panel droit qui offre la possibilité de modifier
un objet, soit par l’utilisation du clavier (Z pour le monter, S pour descendre, Q pour le translater
sur la gauche et D pour une translation vers la droite).

Une fois le niveau adapté, il est possible de le sauvegarder au format \texttt{.lvl}. 
La classe \texttt{MapWriter} a été écrite pour l’occasion, qui permet de sauver 
dans un fichier un niveau, comme \texttt{MapReader} permet de le lire.

Certaines options ont été désactivées dans l’éditeur pour rendre l’édition plus agréable,
notamment la gestion des collisions, les bruits et autres événements spéciaux.

\subsubsection{Faiblesses de l’éditeur}

La gestion des erreurs est totalement absente, ce qui signifie que l’utilisateur est 
responsable des données qu’il entre. Il est tout à fait possible de créer des éléments
dont les caractéristiques ne permettront pas de jouer une partie (une source en dehors
des limites par exemple).

La gestion dynamique des déplacements ne se fait que dans un sens, si l’on déplace un objet
à l’aide du clavier ses propriétés dans le panel de droite seront automatiquement modifiées.
Cependant, il faudra utiliser le bouton « Appliquer » du panel de droite afin que les changements
effectués à cet endroit entrent en application.


\newpage
\section{Présentation des différentes classes}

Dans cette section, nous allons décrire les différentes classes
composants ce projet.
L’implémentation du projet est divisée entre la partie modèle et la partie vue.
Elle est également basée sur le design pattern  Observeur / Observé  comme demandé
dans les consignes.

\subsection{Modèle}

Dans cette section, nous allons décrire les différentes classes du modèle 
(classes métiers).
Un squelette de classes a été fourni par Monsieur Absil.
Ce squelette contenait les fichiers suivants :
`point.h, source.h, dest.h, nuke.h, wall.h,
crystal.h, lens.h, mirror.h, ray.h, level.h`.
Nous avons décidé de modifier ce squelette tout en gardant la 
structure générale.

\subsubsection{Point}

La classe \texttt{Point} représente une position dans un espace à deux dimensions. 
Elle est munie des coordonnées \texttt{x} et \texttt{y} qui sont tous deux des
nombres réels (\texttt{double}). 

La classe \texttt{Point} fourni par Mr. Absil comprenait des coordonnées entières.
Nous avons décidé de passer les coordonnées en nombres réels par soucis de précision.

La classe permet également de calculer la distance entre deux points grâce à la formule :

$$ \sqrt[2]{(x_1 - x_2)^2 + (y_1 - y_2)^2} $$

\subsubsection{Line}

Cette classe représente une droite, elle possède un point et un
angle. À l’aide de ces informations, on peut trouver n'importe quel point de la 
droite en ayant la distance entre le point d'origine
et le point d'arrivée.

Dans les méthodes intersects(...) de la classe Line, un passage
par pointeur de pointeur est fait car un pointeur est passé par 
valeur et lors de l'initialisation il ne pointera donc pas la même
adresse mémoire que le pointeur d'origine.
Si on passait donc par simple pointeur, notre pointeur copié
aurait une bonne zone mémoire mais notre pointeur d'origine
resterait à *nullptr*.

\subsubsection{LineSegment}

Cette classe représente un segment de droite, elle possède
deux points qui représentent les extrémités du segment.

\subsubsection{Ellipse}

Cette classe représente une conique de forme elliptique.
C'est-à-dire une ellipse ou un cercle.

\subsubsection{Rectangle}

Cette classe représente une forme géométrique rectangulaire.
Elle possède un point supérieur gauche ainsi qu'une longueur et hauteur 
nous permettant de retrouver facilement les autres extrémités de la 
forme.

Cette classe sert à représenter les objets Source et Destination.

\subsubsection{Element}

La classe Element est la super-classe de tous les éléments pouvant
se trouver sur une carte de jeu.
Il s'agit de :

\begin{itemize}
    \item \texttt{Source}
    \item \texttt{Dest}
    \item \texttt{Wall}
    \item \texttt{Mirror}
    \item \texttt{Lens}
    \item \texttt{Crystal}
    \item \texttt{Nuke}
\end{itemize}

La classe Ray (cf. Ray) n'est pas un élément, les éléments concernent 
les objets pouvant intérragir avec un rayon.
Il s'agit principalement d'une classe \texttt{tag} pouvant
donner le type de l'élément. Son rôle est de pouvoir retrouver
le type d'un objet lors d'une intersection (cf. Level).

Une énumération fortement typée est donc présente. Elle s'apelle
\texttt{Type} et reprend les noms des éléments cités ci-dessus.
    
\subsubsection{Source}




\subsubsection{Dest}
\subsubsection{Wall}
\subsubsection{Mirror}
\subsubsection{Lens}
\subsubsection{Crystal}
\subsubsection{Nuke}
\subsubsection{Ray}


\subsubsection{Level}

Cette classe représente la classe principale du jeu, elle gère toute 
la logique métier du jeu. Elle permet de calculer la trajectoire
du rayon lumineux.
	
\subsection{Vue}

L’interface graphique a été réalisée en `Qt` à la main.

Les classes composant la partie vue de l’application sont :

\subsubsection{DestinationView}

\paragraph{Description}

Classe modélisant la destination à atteindre par le rayon émis depuis la source pour gagner la partie.

\subsubsection{MapView}

\paragraph{Description}

Classe représentant le plateau de jeu, où tous les éléments sont disposés et affichés.

\subsubsection{MirrorView}

\paragraph{Description}

Classe représentant un miroir sur lequel un rayon peut être réfléchi.

\subsubsection{NukeView}

\paragraph{Description}

Classe représentant une bombe, élément explosif du plateau qui fait perdre la partie
si touché par un rayon lumineux (RayView).

\subsubsection{SourceView}

\paragraph{Description}

Classe représentant une source lumineuse sur le plateau de jeu. Elle dispose de 2 états, allumée ou éteinte.
Passer d’un état à l’autre change l’image la représentant et produit un son d’interrupteur.

\subsubsection{WallView}

\paragraph{Description}

Classe représentant un mur, élément visuel sur lequel le rayon est stoppé.

\section{Conclusion}

\section{Bibliographie}

\section{Annexes}

\subsection{Démarches mathématiques}

\subsubsection{Trouver l'intersection entre deux droites}



\begin{equation} \label{eq:droite}
	D \equiv y = ax + b 
\end{equation}
\begin{equation} \label{eq:droiteverticale}
	D \equiv x = k 
\end{equation}


\eqref {eq:droite} 
représente une droite non verticale.
\eqref {eq:droiteverticale} 
représente une droite verticale où k est une valeur
quelquonque sur l'axe des x.

Dans un premier temps, il faut tester que la pente $a$ est différente. 
Si cette dernière est égale, il n'y a pas d'intersections. En effet,
si les pentes sont les mêmes, cela signifie que les droites sont soit
parallèles soit égales. Dans le cas de droites égales, on considère qu'il
n'y a pas d'intersections, bien qu'en réalité il y en a une infinité.

Si la pente est différente,
l'intersection entre deux droites est assez simple, il suffit 
d'injecter une variable d'une des deux équations dans l'autre.
Dans le cadre du projet, il faut evidemment faire attention aux 
droites verticales.

Le cas d'une droite verticale : 

On injecte \eqref{eq:droiteverticale} dans \eqref{eq:droite}.

\begin{equation} \label{eq:interv}
	D \equiv y = ak + b
\end{equation}

Avec \eqref{eq:interv} on obtient $y$ du point dont le $x = k$
pour une droite verticale \eqref{eq:droiteverticale}.

Pour deux droites non verticales on injecte le $y$ d'une équation
de type \eqref{eq:droite} dans le $y$ d'une autre équation de typ \eqref{eq:droite}.

\begin{equation} \label{eq:internv}
	a remplir
\end{equation}

$ x = \frac {(y - b)}{ a} $ pour une droite non verticale
dont le y est celui obtenu plus tôt \eqref{eq:droite}.

\subsubsection{Trouver l'intersection entre une droite et un segment}

\subsubsection{Trouver le(s) intersections entre une ellipse et une droite}

La formule d'une ellipse est : 

$$ E \equiv \frac{(x - x1)^2}{a^2} + \frac{(y - y1)^2}{b^2} = 1 $$

\begin{description}
\item où $x1$ et $y1$ sont respectivement les coordonnées
$x$ et $y$ du centre de l'ellipse. \\
\item où $a$ et $b$ sont respectement les rayons de l'axe
$x$ et $y$. \\
\end{description}

Pour trouver une intersection entre une ellipse
et une droite, il faut égaler deux variables 
identiques :


On doit donc remplacer la variable $x$ ou $y$ de la droite dans
l'équation de l'ellipse.

Le cas de la droite verticale : 

Dans ce cas la, il n'y a pas de choix, il faut remplacer
$x$ dans l'équation de l'ellipse.
Nous avons également décidé de refactoriser l'équation en 
prenant le PPCM (Plus petit commun multiple) de $a^2 * b^2$ que
nous apellerons ici $lcm$ pour Least Commun Multiple.
Ce choix a été fait pour éviter les overflows lorsque 
de nombres trop grands sont mis au carré et multipliés.
Bien que dans notre cas, on nous avons rarement des nombre
pouvant obtenir un tel résultat.

$$ E \equiv (lcmy \cdot (k - x1)^2) + ((y - y1)^2 \cdot lcmx) = lcm $$  
où $lcmy$ est le facteur par lequel il faut multiplier
$b^2$ (rayon $y$ au carré) pour obtenir $lcm$, \\
où $lcmx$ est le facteur par lequel il faut multiplier
$a^2$ (rayon $x$ au carré) pour obtenir $lcm$.

$$ E \equiv lcmy \cdot (k-x1)^2 + (y^2 + y1^2 - 2 \cdot y1 \cdot y)
\cdot lcmx = lcm$$
$$ E \equiv lcmy \cdot (k-x1)^2 + lcmx \cdot y^2 + lcmx \cdot
y1^2 - 2 \cdot lcmx \cdot y1 \cdot y - lcm = 0 $$

Avec ceci, il reste plus qu'à résoudre l'équation du second degré 
avec 
$$ \rho = b^2 - 4ac$$
où $$ a = lcmx $$
$$ b = 2 \cdot lcmx \cdot y1 \cdot y $$
$$ c = (lcmy \cdot (k-x1)^2) + (lcmx \cdot y1^2) - lcm $$



Le nombre d'intersections est différent selon la valeur de $\rho$.

\[
	n =
	\begin{cases}
		0 & \text{si } \rho < 0 \\
		1 & \text{si } \rho = 0 \\
		2 & \text{si } \rho > 0  
	\end{cases}
\]

$$ y = \frac {-b}{2a} $$

$$ y1 = \frac{(-b + \sqrt{\rho})}{2a} $$
$$ y2 = \frac{(-b - \sqrt{\rho})}{2a} $$

On a donc le(s) $y$ du/des points d'intersections, et le $x$
vaut $k$ (de l'équation de départ).



Le cas de la droite non verticale : 


C'est le même principe que le cas de la droite verticale sauf
que pour trouver la deuxième variable finale il faudra
remplacer la variable trouvée dans l'équation de la droite.

\subsubsection{Trouver le(s) intersections entre une ellipse et un segment}

Il s'agit du même principe que les intersections droite/segment.
Il faut vérifier les intersections entre les ellipses et un segment
transformé en droite et ensuite vérifier si les points d'intersections
se trouvent sur le segment.

\end{document}
