\documentclass[]{article}
\usepackage{amssymb,amsmath}

\usepackage[utf8]{inputenc}
\usepackage[francais]{babel}

\usepackage[unicode=true,
	colorlinks=true,
linkcolor=blue]{hyperref}

\hypersetup{breaklinks=true, pdfborder={0 0 0}}
\setlength{\parindent}{0pt}
\setlength{\parskip}{6pt plus 2pt minus 1pt}
\setlength{\emergencystretch}{3em}  % prevent overfull lines
\setcounter{secnumdepth}{0}

\title{Rapport du projet Starlight}
\author{Florian Knop (39310) - Gatien Bovyn (39189)}

\begin{document}
\maketitle

\newpage

\tableofcontents

\newpage

\section{Introduction}


Ce document vise à présenter le travail d’analyse et de programmation effectué
lors de la réalisation du projet du laboratoire Langage C++ : Starlight.

Ce projet a été réalisé en binôme par Florian Knop, matricule 39310 groupe 2G13,
et Gatien Bovyn, matricule 39189 groupe 2G11.

Le programme à concevoir consiste en une implémentation du modèle et d’une interface
graphique du jeu baptisé Starlight, puzzle à 2 dimensions basé sur la lumière.


\section{Sections}

\section{Présentation des différentes classes}


L’implémentation du projet est divisée entre la partie modèle et la partie vue.
Elle est également basée sur le design pattern  Observeur / Observé  comme demandé
dans les consignes.

\subsection{Modèle}


Un squelette de l’application nous a été fourni par Monsieur Absil. Ce squelette contient les fichiers suivants :

`point.h, source.h, dest.h, nuke.h, wall.h, crystal.h, lens.h, mirror.h, ray.h, level.h`.

\subsubsection{Line}


Cette classe représente une droite, elle possède un point et un
angle. Gràce à cela on peut trouver n'importe quel point de la 
droite gràce en ayant la distance entre le point d'origine
et le point d'arrivée.

Dans les méthodes intersects(...) de la classe Line, un passage
par pointeur de pointeur est fait car un pointeur est passé par 
valeur et lors de l'initialisation il ne pointera donc pas la même
adresse mémoire que le pointeur d'origine.
Si on passait donc par simple pointeur, notre pointeur copié
aurait une bonne zone mémoire mais notre pointeur d'origine
resterait à *nullptr*.

\subsubsection{Ellipse}


Cette classe représente une conique de forme elliptique.
C'est à dire une ellipse ou un cercle.


\subsection{Vue}


L’interface graphique a été réalisée en `Qt` à la main.

Les classes composant la partie vue de l’application sont :

\subsubsection{DestinationView}

\paragraph{Description}


Classe modélisant la destination à atteindre par le rayon émis depuis la source pour gagner la partie.

\paragraph{MapView}

\paragraph{MirrorView}

\paragraph{NukeView}

\paragraph{SourceView}

\paragraph{WallView}

\section{Conclusion}

\section{Bibliographie}

\section{Annexes}

\subsection{Démarches mathématique}

\subsubsection{Trouver l'intersection entre deux droites}



\begin{equation} \label{eq:droite}
	D \equiv y = ax + b 
\end{equation}
\begin{equation} \label{eq:droiteverticale}
	D \equiv x = k 
\end{equation}


\eqref {eq:droite} 
représente une droite non verticale.
\eqref {eq:droiteverticale} 
représente une droite verticale où k est une valeur
quelquonque sur l'axe des x.

Dans un premier temps, il faut tester que la pente $a$ est différente. 
Si cette dernière est égale, il n'y a pas d'intersections. En effet,
si les pentes sont les mêmes, cela signifie que les droites sont soit
parallèles soit égales. Dans le cas de droites égales, on considère qu'il
n'y a pas d'intersections, bien qu'en réalité il y en a une infinité.

Si la pente est différente,
l'intersection entre deux droites est assez simple, il suffit 
d'injecter une variable d'une des deux équations dans l'autre.
Dans le cadre du projet, il faut evidemment faire attention aux 
droites verticales.

Le cas d'une droite verticale : 

On injecte \eqref{eq:droiteverticale} dans \eqref{eq:droite}.

\begin{equation} \label{eq:interv}
	D \equiv y = ak + b
\end{equation}

Avec \eqref{eq:interv} on obtient $y$ du point dont le $x = k$
pour une droite verticale \eqref{eq:droiteverticale}.

Pour deux droites non verticales on injecte le $y$ d'une équation
de type \eqref{eq:droite} dans le $y$ d'une autre équation de typ \eqref{eq:droite}.

\begin{equation} \label{eq:internv}
	a remplir
\end{equation}

$ x = \frac {(y - b)}{ a} $ pour une droite non verticale
dont le y est celui obtenu plus tôt \eqref{eq:droite}.

\subsubsection{Trouver l'intersection entre une droite et un segment}

\subsubsection{Trouver le(s) intersections entre une ellipse et une droite}

La formule d'une ellipse est : 

$$ E \equiv \frac{(x - x1)^2}{a^2} + \frac{(y - y1)^2}{b^2} = 1 $$

\begin{description}
\item où $x1$ et $y1$ sont respectivement les coordonnées
$x$ et $y$ du centre de l'ellipse. \\
\item où $a$ et $b$ sont respectement les rayons de l'axe
$x$ et $y$. \\
\end{description}

Pour trouver une intersection entre une ellipse
et une droite, il faut égaler deux variables 
identiques :


On doit donc remplacer la variable $x$ ou $y$ de la droite dans
l'équation de l'ellipse.

Le cas de la droite verticale : 

Dans ce cas la, il n'y a pas de choix, il faut remplacer
$x$ dans l'équation de l'ellipse.
Nous avons également décidé de refactoriser l'équation en 
prenant le PPCM (Plus petit commun multiple) de $a^2 * b^2$ que
nous apellerons ici $lcm$ pour Least Commun Multiple.
Ce choix a été fait pour éviter les overflows lorsque 
de nombres trop grands sont mis au carré et multipliés.
Bien que dans notre cas, on nous avons rarement des nombre
pouvant obtenir un tel résultat.

$$ E \equiv (lcmy \cdot (k - x1)^2) + ((y - y1)^2 \cdot lcmx) = lcm $$  
où $lcmy$ est le facteur par lequel il faut multiplier
$b^2$ (rayon $y$ au carré) pour obtenir $lcm$, \\
où $lcmx$ est le facteur par lequel il faut multiplier
$a^2$ (rayon $x$ au carré) pour obtenir $lcm$.

$$ E \equiv lcmy \cdot (k-x1)^2 + (y^2 + y1^2 - 2 \cdot y1 \cdot y)
\cdot lcmx = lcm$$
$$ E \equiv lcmy \cdot (k-x1)^2 + lcmx \cdot y^2 + lcmx \cdot
y1^2 - 2 \cdot lcmx \cdot y1 \cdot y - lcm = 0 $$

Avec ceci, il reste plus qu'à résoudre l'équation du second degré 
avec 
$$ \rho = b^2 - 4ac$$
où $$ a = lcmx $$
$$ b = 2 \cdot lcmx \cdot y1 \cdot y $$
$$ c = (lcmy \cdot (k-x1)^2) + (lcmx \cdot y1^2) - lcm $$



Le nombre d'intersections est différent selon la valeur de $\rho$.

\[
	n =
	\begin{cases}
		0 & \text{si } \rho < 0 \\
		1 & \text{si } \rho = 0 \\
		2 & \text{si } \rho > 0  
	\end{cases}
\]

$$ y = \frac {-b}{2a} $$

$$ y1 = \frac{(-b + \sqrt{\rho})}{2a} $$
$$ y2 = \frac{(-b - \sqrt{\rho})}{2a} $$

On a donc le(s) $y$ du/des points d'intersections, et le $x$
vaut $k$ (de l'équation de départ).



Le cas de la droite non verticale : 


C'est le même principe que le cas de la droite verticale sauf
que pour trouver la deuxième variable finale il faudra
remplacer la variable trouvée dans l'équation de la droite.

\subsubsection{Trouver le(s) intersections entre une ellipse et un segment}

\end{document}
