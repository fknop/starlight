\documentclass[]{article}
\usepackage{amssymb,amsmath}
\usepackage{amsthm}
\usepackage[utf8]{inputenc}
\usepackage[francais]{babel}

\usepackage[unicode=true,
	colorlinks=true,
linkcolor=blue]{hyperref}

\hypersetup{breaklinks=true, pdfborder={0 0 0}}
\setlength{\parindent}{0pt}
\setlength{\parskip}{6pt plus 2pt minus 1pt}
\setlength{\emergencystretch}{3em}  % prevent overfull lines
\setcounter{secnumdepth}{0}

\title{Rapport du projet Starlight}
\author{Florian Knop (39310) - Gatien Bovyn (39189)}

\begin{document}
\maketitle

\newpage

\tableofcontents

\newpage

\section{Introduction}


Ce document vise à présenter le travail d’analyse et de programmation effectué
lors de la réalisation du projet du laboratoire Langage C++~:~Starlight.

Ce projet a été réalisé en binôme par Florian Knop, matricule 39310 groupe 2G13,
et Gatien Bovyn, matricule 39189 groupe 2G11.

Le programme à concevoir consiste en une implémentation du modèle et d’une interface
graphique du jeu baptisé Starlight, puzzle à 2 dimensions basé sur la lumière.

Ce projet a été compilé principalement avec g++
(version 4.8.2 ou supérieure)
sous la distribution Linux Ubuntu (ou une de ses dérivées).
La version du framework Qt utilisée est la 5.0.2 ou supérieure.
Ce projet a été fait sous QtCreator, IDE OpenSource en version 2.8.1 ou
supérieure.

\section{Sections}

\section{Conventions de nommages utilisées}

Dans cette section, nous présenterons les différentes conventions 
utilisées lors de ce projet. Nous avons décidé de reprendre certaines
conventions utilisées par la STL (Librairie Standard)
tout en utilisant d'autres conventions.

De manière globale, tous les noms de variables, classes, fichiers, etc.
sont en anglais.

\subsection{Nommage des fichiers}

Les noms de fichiers sont entièrement en minuscules
et possèdent les extensions .h
pour les headers et .cpp pour les fichiers sources.

\begin{itemize}
	\item \texttt{nomfichier.h}
	\item \texttt{nomfichier.cpp}
\end{itemize}

\subsection{Classes}

Les noms des classes commencent par une majuscule à chaque mot.
Cette convention est également valable pour les noms 
d'énumérations et structures.

\begin{itemize}
	\item \texttt{Classe}
	\item \texttt{NomClasse}
	\item \texttt{NomStruct}
	\item \texttt{NomEnumeration}
\end{itemize}

\newpage

\subsection{Variables}

\subsubsection{Variables de classe}

Les noms des variables de classes sont en minuscules, les mots sont 
séparés par un underscore et le nom est suffixé par un underscore.

\begin{itemize}
	\item \texttt{variable\_}
	\item \texttt{nom\_variable\_}
\end{itemize}

Cette convention est également valable pour les variables d'une
structure.

\subsubsection{Variables locales}

Les noms des variables locales respectent les mêmes conventions 
que les variables de classe, à la seule exception qu’ils ne sont pas
suffixés par un underscore.

\begin{itemize}
	\item \texttt{variable}
	\item \texttt{nom\_variable}
\end{itemize}

\subsubsection{Constantes}

Les noms des constantes sont entièrement en majuscules et les mots sont
séparés par des underscores.

\begin{itemize}
	\item \texttt{CONSTANTE}
	\item \texttt{NOM\_CONSTANTE}
\end{itemize}

\subsection{Méthodes}

D'une manière générale, les noms de méthodes sont en minuscules et
séparés par des underscores.

\subsubsection{Getters}

Le nom d'un getter (accesseur en lecture de variable de classe/structure)
est égal au nom de la variable de classe sans l'underscore final.

\begin{itemize}
	\item Variable : \texttt{variable\_} - Getter : \texttt{variable()}
	\item Variable : \texttt{nom\_var\_} - Getter : \texttt{nom\_var()}
\end{itemize}

\subsubsection{Setters}

Le nom d'un setter (accesseur en écriture de variable de classe/structure)
est égal au nom de la variable sans l'underscore final préfixé de \texttt{set}.

\begin{itemize}
	\item Variable : \texttt{variable\_} - Setter : \texttt{set\_variable()}
	\item Variable : \texttt{nom\_var\_} - Setter : \texttt{set\_nom\_var()}
\end{itemize}

\subsubsection{Autres méthodes}

Les autres méthodes possèdent les mêmes conventions que celles
énoncées ci-dessus.

\begin{itemize}
	\item \texttt{methode()}
	\item \texttt{nom\_methode()}
\end{itemize}

\subsection{Éléments d'une énumération} 

Les éléments d'une énumération suivent les mêmes conventions que celles des constantes énoncées ci-dessus.

\newpage
\section{Présentation générale du projet}

Cette section décrit les objectifs principaux et secondaires effectués lors
de la réalisation de ce projet.
Toutes les classes utilisées sont décrites dans la section suivante.

Le projet fini ouvre sur un menu principal permettant
de jouer à Starlight, de voir les rêgles du jeu, d'accéder à l'éditeur de 
carte ou de simplement quitter.

\subsection{Le jeu}

Le projet permet de jouer à Starlight en important sa propre carte de jeu
au format \texttt{.lvl} qu'il est possible de créer soi-même grâce à l'éditeur
de carte (cf. \hyperref[Editeur]{''Editeur de cartes''}).

Il est nécessaire de préciser que le jeu part du principe que la carte 
fournie est sans erreurs. Une carte fournie avec erreur produira donc
un arrêt immédiat de l'application.

\subsubsection{Le fonctionnement du jeu}

Le but du jeu est de déplacer un rayon provenant d'une source lumineuse
pouvant être allumée ou éteinte
vers une destination à l'aide de miroirs plans amovibles réfléchissant
la lumière.
La lumière possède une longueur d'onde comprise dans le spectre
de lumière visible.

En addition avec leur déplacement, les miroirs peuvent tourner
autour d'un point de pivot.
Ce déplacement et cette rotation se font tout en restant
dans certaines limites si celles-ci ont été définies à la création du miroir.

Une gestion des collisions a également été implémentée pour
éviter que les miroirs déplacés ne heurtent les autres éléments de la carte.
De ce fait, la carte de jeu fournie par Mr. Absil a légèrement été modifiée
pour éviter que les miroirs ne se trouvent dans les murs à la création du niveau
et qu'il soit donc impossible de les déplacer par la suite.

La carte peut aussi posséder des cristaux qui modifient la longueur d'onde
du rayon les traversant, ainsi que des lentilles laissant passer la lumière
d'une certaine intervalle de longueur d'onde.
La couleur de la lumière est donc modifiée selon la longueur d'onde 
du rayon lumineux.

Et pour finir, la carte peut être munie de bombes qui terminent et font perdre
instantanément la partie si celles-ci sont touchées. 


\subsubsection{L'interface}

L'interface graphique du jeu Starlight a été réalisée sous Qt (en version 5.0.2 ou supérieure).
Il s'agit ici d'une interface simple et minimaliste permettant d'effectuer la fonction de base 
demandée : jouer.

La fenêtre de jeu possède également des menus permettants de :

\begin{itemize}
    \item \texttt{Quitter}
    \item \texttt{Revenir au menu principal}
    \item \texttt{Charger une carte}
    \item \texttt{Quitter la carte}
    \item \texttt{Voir les règles du jeu}
\end{itemize}

Les raccourcis clavier permettant d'accéder aux fonctions citées ci-dessus sont dépendants
du système d'exploitation utilisé.

Les miroirs sont sélectionnés en utilisant un double clic gauche.
Si un troisième clic s’ensuit, les limites de déplacement du pivot s'affichent
sous la forme d'un rectangle bleu.

Les miroirs peuvent être déplacés et tournés au clavier en utilisant les touches~:

\begin{itemize}
	\item \texttt{Z} pour déplacer vers le haut
	\item \texttt{S} pour déplacer vers le bas
	\item \texttt{Q} pour déplacer vers la gauche
	\item \texttt{D} pour déplacer vers la droite
	\item \texttt{Flèche directionnelle gauche} pour tourner dans le sens
		anti-horloger
	\item \texttt{Flèche directionnelle droite} pour tourner dans le sens
		horloger
    \item \texttt{Shift} + une des touches citées ci-dessus pour se déplacer / tourner
        plus vite
\end{itemize}

Des sons ont été ajoutés lorsque la source est allumée ou éteinte, 
lorsqu'une bombe a été touchée et lorsque la destination
est atteinte.

\paragraph{Faiblesses de l'interface} 

L'interface ne permet pas de déplacer les miroirs à la souris, mais seulement au clavier.
Le rectangle bleu affichant les limites du miroir s'affiche après un 3e clic plutôt 
que lors de la sélection du miroir.

Les sons ajoutés qui ont été cités ci-dessus ne s'activent pas toujours.
Par exemple lorsque la source est allumée et éteinte très vite.
Il se peut également que la source n'émette simplement pas de son.


\subsection{\label{Editeur}Éditeur de cartes}

L’éditeur de carte a été conçu sur base du modèle et des vues existantes.
Il reprend tous les éléments disponibles dans un niveau du jeu Starlight.

\subsubsection{Possibilités de l’éditeur}

Au lancement de l’éditeur, il est possible de charger un niveau existant 
ou d’en créer un nouveau en personnalisant sa taille. Celle-ci ne peut 
être changée par la suite.

Une fois chargé/créé, il est alors possible d’ajouter de nouveaux éléments 
dans le niveau (mur, miroir, lentille, cristal, bombe). Comme prévu par l’énoncé,
une seule source et une seule destination peuvent être incluses. Dès lors, 
celles-ci sont créées en même temps que le niveau et il n’est pas possible de 
les supprimer. Il n'est également pas possible de supprimer les murs extérieurs
au niveau.

Tous les éléments ont la possibilité d’être modifiés à tous niveaux, les setters
appropriés ayant été ajoutés dans les classes du modèle. Il est au minimum possible de déplacer
tous les éléments, soit par le panel droit qui offre la possibilité de modifier
un objet, soit par l’utilisation du clavier (Z pour le monter, S pour descendre, Q pour le translater
sur la gauche et D pour une translation vers la droite).

Une fois le niveau adapté, il est possible de le sauvegarder au format \texttt{.lvl}. 
La classe \texttt{MapWriter} a été écrite pour l’occasion, qui permet de sauver 
dans un fichier un niveau, comme \texttt{MapReader} permet de le lire.

Certaines options ont été désactivées dans l’éditeur pour rendre l’édition plus agréable,
notamment la gestion des collisions, les bruits et autres événements spéciaux.

\subsubsection{Faiblesses de l’éditeur}

La gestion des erreurs est totalement absente, ce qui signifie que l’utilisateur est 
responsable des données qu’il entre. Il est tout à fait possible de créer des éléments
dont les caractéristiques ne permettront pas de jouer une partie (une source en dehors
des limites par exemple).

La gestion dynamique des déplacements ne se fait que dans un sens, si l’on déplace un objet
à l’aide du clavier ses propriétés dans le panel de droite seront automatiquement modifiées.
Cependant, il faudra utiliser le bouton « Appliquer » du panel de droite afin que les changements
effectués à cet endroit entrent en application.


\newpage
\section{Présentation des différentes classes}

Dans cette section, nous allons décrire les différentes classes
composant ce projet.
L’implémentation du projet est divisée entre la partie modèle et la partie vue ainsi 
qu'une partie classes utilitaires.
Elle est également basée sur le design pattern «~Observeur~/~Observé~» comme demandé
dans les consignes. 


\subsection{Les classes/namespaces utilitaires}

\subsubsection{MapReader}

La classe \texttt{MapReader} est la classe qui lit un fichier \texttt{.lvl} 
et crée le niveau (cf. \hyperref[Level]{''Level''}).

Cette classe considère que le fichier est sans erreur. Une instantiation d'objets
erronés produira donc un arrêt de l'application.

Cette classe est basée sur le \texttt{Singleton Pattern}. On ne peut qu’instancier un niveau
à la fois.

\subsubsection{MapWriter}

La classe \texttt{MapWriter} est la classe qui va écrire un niveau (cf. \hyperref[Level]{''Level''}) 
dans un fichier texte. Il s’agit donc du processus inverse de la classe
\texttt{MapReader}. Cette classe sert à l’éditeur lors de la sauvegarde de la
carte en cours d’édition.

\subsubsection{\label{Constants}Constants}

\texttt{constants.h} est un header reprenant toutes les constantes utilisées
dans ce projet. 

\begin{itemize}
    \item \texttt{INF} correspond à l'infini.
    \item \texttt{EPSILON} correspond à \texttt{0.00001}. EPSILON permet
        d'éviter les imprécisions entre deux nombres réels (cf. \hyperref[equals]{''umath - méthode equals()''}).
    \item \texttt{PI} correspond à la valeur de PI \texttt{3.14159...} avec
        autant de décimales possibles qu'un \texttt{double} peut contenir.
    \item \texttt{PI\_2} correspond à la valeur de PI divisé par deux.
    \item \texttt{PI\_4} correspond à la valeur de PI divisé par quatre.
    \item \texttt{PI\_2\_3} correspond à la valeur de PI\_2 multiplié par trois.
\end{itemize}

\subsubsection{Le namespace umath}

Le namespace \texttt{umath} possède toutes les méthodes utilitaires mathématiques
et géométriques servant au projet. Comme par exemple des méthodes permettant de trouver
les intersections entre droites, segments, ellipses et rectangles.

\label{equals}
\texttt{umath} reprend des méthodes d'égalité de nombre réels. Deux nombres réels sont égaux
si la valeur absolue de la soustraction de ceux-ci est plus petite que la valeur de EPSILON (cf. \hyperref[Constants]{''Constants''}).

Une autre méthode d’égalité permet de vérifier si un nombre équivaut à l’infini (INF ou -INF).
Il y a également des méthodes de conversions entre pentes, radians, degrés, etc.

Les méthodes d’intersections sont décrites plus en détails dans la section \hyperref[Annexe]{Annexe - Démarches mathématiques}.

\subsection{Modèle}

Dans cette section, nous allons décrire les différentes classes du modèle 
(classes métiers).
Un squelette de classes a été fourni par Monsieur Absil.
Ce squelette contenait les fichiers suivants~:
`point.h, source.h, dest.h, nuke.h, wall.h,
crystal.h, lens.h, mirror.h, ray.h, level.h` ainsi que leurs sources correspondantes.
Nous avons décidé de modifier ce squelette tout en gardant la 
structure générale.

\subsubsection{\label{Point}Point}

La classe \texttt{Point} représente une position dans un espace à deux dimensions. 
Elle est munie des coordonnées \texttt{x} et \texttt{y} qui sont tous deux des
nombres réels (\texttt{double}). 

La classe \texttt{Point} fourni par Mr. Absil comprenait des coordonnées entières.
Nous avons décidé de passer les coordonnées en nombres réels par souci de précision.

La classe permet également de calculer la distance entre deux points grâce à la formule~:

$$ \sqrt[2]{(x_1 - x_2)^2 + (y_1 - y_2)^2} $$

\subsubsection{Line}

La classe \texttt{Line} représente une droite de la forme~: 
$ D \equiv ax + by + c = 0 $ \\

$ D \equiv ax + c = 0    (b = 0) $ représente l'équation d'une droite verticale. \\
$ D \equiv by + c = 0    (a = 0) $ représente l'équation d'une droite horizontale. \\

Elle possède les trois paramètres $ a $, $ b $ et $ c $ ainsi que l'angle
que forme la droite pour éviter de devoir reconvertir la pente calculée par $ \frac{-a}{b} $
en angle à chaque fois.

La classe \texttt{Line} possède des méthodes permettant de savoir si celle-ci est 
verticale ($ b = 0 $) ou horizontale ($ a = 0 $), si elle est perpendiculaire ou parallèle à une autre droite.
Des méthodes permettant d'obtenir $ x $ selon une valeur de $ y $ donné et inversément sont également
présentes. Cependant ces méthodes peuvent renvoyer une valeur infinie dans le cas où la droite 
est soit verticale soit horizontale.

Cette classe sert principalement à modéliser un rayon de lumière (cf. \hyperref[Ray]{''Ray''}) pour trouver 
les intersections entre le rayon et les éléments du jeu.

\subsubsection{LineSegment}

La classe \texttt{LineSegment} représente un segment de droite possédant
deux points (cf. \hyperref[Point]{''Point''}) qui sont les extrémités du segment.

Nous avons fait le choix qu'un segment puisse posséder deux fois le même point.
Bien que cela n'ait pas vraiment de sens purement mathématique, un segment
ayant deux fois le même point est tout simplement un point.

Le segment peut être transformé en droite grâce aux deux points
le constituant et ainsi former l'équation de droite.

Le segment peut également être décalé d'un certain $ x $ et $ y $ ainsi que tourné 
d'un certain angle\footnote{\hyperref[AnnexeRotation]{''Rotation de segment''}}.



\subsubsection{Ellipse}

Cette classe représente une conique de forme elliptique.
C'est-à-dire une ellipse ou un cercle.

\subsubsection{Rectangle}

Cette classe représente une forme géométrique rectangulaire.
Elle possède un point supérieur gauche ainsi qu'une longueur et hauteur 
nous permettant de retrouver facilement les autres extrémités de la 
forme.

Cette classe sert à représenter les objets Source et Destination.

\subsubsection{Element}

La classe Element est la super-classe de tous les éléments pouvant
se trouver sur une carte de jeu.
Il s'agit de :

\begin{itemize}
    \item \texttt{Source}
    \item \texttt{Dest}
    \item \texttt{Wall}
    \item \texttt{Mirror}
    \item \texttt{Lens}
    \item \texttt{Crystal}
    \item \texttt{Nuke}
\end{itemize}

La classe Ray (cf. \hyperref[Ray]{''Ray''}) n’est pas un élément, les éléments concernent 
les objets pouvant interagir avec un rayon.
Il s'agit principalement d'une classe \texttt{tag} pouvant
donner le type de l'élément. Son rôle est de pouvoir retrouver
le type d'un objet lors d'une intersection (cf. \hyperref[Level]{''Level''}).

Une énumération fortement typée est donc présente. Elle s'appelle
\texttt{Type} et reprend les noms des éléments cités ci-dessus.
    
\subsubsection{Source}




\subsubsection{Dest}
\subsubsection{Wall}
\subsubsection{Mirror}
\subsubsection{Lens}
\subsubsection{Crystal}
\subsubsection{Nuke}
\subsubsection{\label{Ray}Ray}


\subsubsection{\label{Level}Level}

Cette classe représente la classe principale du jeu, elle gère toute 
la logique métier du jeu. Elle permet de calculer la trajectoire
du rayon lumineux.
	
\subsection{Vue}

L’interface graphique a été réalisée en `Qt` à la main. Chaque élément visuel dispose d’un pointeur vers son
équivalent dans le modèle, sur base duquel il est construit. Il observe également cet élément afin de se mettre
à jour automatiquement.

Chaque élément a également été prévu pour être utilisé dans l’éditeur, et dispose ainsi de booléens pour les rendre 
sélectionnables quand cela est nécessaire.

Les classes composant la partie vue de l’application sont :

\subsubsection{CrystalView}

Classe modélisant un cristal, élément oval qui modifie la longueur d’onde d’un rayon si celui-ci traverse ledit cristal.

\subsubsection{DestinationView}

Classe modélisant la destination à atteindre par le rayon émis depuis la source pour gagner la partie.

\subsubsection{ElementView}

Classe servant de super-classe à toutes les éléments présents sur un \texttt{MapView}.
Il s'agit de :

\begin{itemize}
    \item \texttt{SourceView}
    \item \texttt{DestinationView}
    \item \texttt{WallView}
    \item \texttt{MirrorView}
    \item \texttt{LensView}
    \item \texttt{CrystalView}
    \item \texttt{NukeView}
\end{itemize}

\subsubsection{LensView}

Classe modélisant une lentille à travers laquelle un rayon peut passer si sa longueur d’onde est
comprise entre les valeurs de la lentille. Sinon la lentille se comporte comme un mur et le rayon 
est stoppé.

\subsubsection{MapView}

Classe représentant le plateau de jeu, où tous les éléments sont disposés et affichés.
Elle s’occupe de gérer les raccourcis clavier utilisés pour déplacer/pivoter les miroirs dans le jeu et de déplacer les éléments en général dans l’éditeur de carte.

\subsubsection{MirrorView}

Classe représentant un miroir sur lequel un rayon peut être réfléchi.

\subsubsection{NukeView}

Classe représentant une bombe, élément explosif du plateau qui fait perdre la partie
si touché par un rayon lumineux (\texttt{RayView}). Dès le moment où elle est illuminée, une bombe
change de couleur (passe du noir au rouge) et produit un son et un message visuel indiquant la fin de partie.

\subsubsection{RayView}

Classe modélisant un rayon lumineux, émis depuis la source (\texttt{SourceView}) et destiné à atteindre la destination 
(\texttt{DestView}) en se reflétant sur des miroirs et en passant à travers des cristaux si nécessaire.
Un \texttt{RayView} a une couleur différente selon sa longueur d’onde comme proposé dans les objectifs secondaires,
cette couleur reflétant la couleur réelle qu’aurait un rayon lumineux de cette longueur d’onde.

\subsubsection{SourceView}

Classe représentant une source lumineuse sur le plateau de jeu. Bien que dans les consignes, la source lumineuse soit un carré, nous avons ensuite utilisé des images pour la représenter.
La source dispose de 2 états, allumée ou éteinte. Passer d’un état à l’autre change l’image la représentant et produit un son d’interrupteur.

\subsubsection{WallView}

Classe représentant un mur, élément visuel sur lequel le rayon est stoppé.

\section{Conclusion}

\section{Bibliographie}

Les sons utilisés ont été produits par Mike Koenig et sont sous licence Attribution 3.0.\\

Bruit de bombe~:~http://soundbible.com/106-Car-Explosion.html \\
Bruit de victoire~:~http://soundbible.com/1003-Ta-Da.html \\
Bruit d’interrupteur~:~http://soundbible.com/761-Switch.html \\

\section{\label{Annexe}Annexes}

\subsection{Annexe A : Démarches mathématiques}

\subsubsection{\label{AnnexeRotation}Rotation de segment}

La méthode de rotation de segment utilisée se base sur les formules
de coordonnées polaires\footnote{\href{http://fr.wikipedia.org/wiki/Coordonn\%C3\%A9es\_polaires}{Wikipedia : Coordonnées polaires}}
ainsi que les formules trigonométriques d'additions
\footnote{\href{http://fr.wikipedia.org/wiki/Trigonom\%C3\%A9trie}{Wikipedia : Formules trigonométriques}}.

Pour tourner un segment, il faut un point de pivot. Ce point de pivot
doit ensuite être déplacé sur l'origine du repère $ (0, 0) $.

Les formules de coordonnées polaires pour $ x $ et $ y $ sont :
$$ x = r * \cos{q} $$ 
$$ y = r * \sin{q} $$

$ r $ correspond à la distance entre l'origine et le point et $ q $ correspond 
à l'angle entre l'axe des abscisses et la droite formée avec l'origine et le point.

Lorsque l'on souhaite tourner un segment et trouver $ x' $ et $ y' $, 
il suffit d'augmenter ou diminuer l'angle $ q $ par un angle $ f $ comme ceci :

$$ x' = r * \cos{q + f} $$
$$ y' = r * \sin{q + f} $$

Grâce aux formules trigonométriques d'additions, on peut transformer ce résultat en :

$$ x' = r * \cos{q} * \cos{f} - r * \sin{q} * \sin{f} $$
$$ y' = r * \sin{q} * \cos{f} - r * \cos{q} * \sin{f} $$

Et pour finir, en remplaçant les deux premières égalités dans cette dernière on obtient : 

$$ x' = x * \cos{f} - y * \sin{f} $$
$$ y' = x * \sin{f} + y * \cos{f} $$

Il faut ensuite redéplacer le pivot à son point de départ.

\newpage
\subsubsection{Trouver l'intersection entre deux droites}

Une droite possède une intersection avec une autre droite si :
\begin{itemize}
    \item Celles-ci sont parallèles et confondues, alors l'intersection est la droite elle même.
    \item Celles-ci sont non parallèles, alors l'intersection est un point. \\
\end{itemize} 

$ D \equiv ax + by + c = 0 $ représente l'équation d'une droite. \\
$ D \equiv ax + c = 0 $ représente l'équation d'une droite verticale ($b = 0$). \\
$ D \equiv by + c = 0 $ représente l'équation d'une droite horizontale ($a = 0$). \\

On commence donc dans un premier temps à tester si les deux droites son parallèles. \\

Soit deux droites $D_1$ et $D_2$, ces deux droites sont parallèles si : 
$$ a_1 * b_2 - a_2 * b_1 = 0 $$

\begin{proof}
La pente d'une droite vaut $\frac{-a}{b}$ et deux droites sont parallèles
si leur pentes sont égales. Il suffit donc de modifier l'équation en passant chaque terme du bon côté. \\
$ a_1 * b_2 = a_2 * b_1 $ puis ensuite : $ \frac{a_1}{b_1} = \frac{a_2}{b_2} $ ce qui équivaut à
$ \frac{-a_1}{b1} = \frac{-a_2}{b_2} $ \\
\end{proof}

Si les deux droites sont parallèles : 
\begin{itemize}
    \item Les deux droites sont verticales : si leurs $x$ sont égaux, elles sont confondues.
    \item Les deux droites sont horizontales : si leurs $y$ sont égaux, elles sont confondues.
    \item Les droites ne sont ni verticales, ni horizontales, il faut alors tester si les deux
        ordonnées à l'origine sont égales. Cette dernière vaut $\frac{-c}{b}$. \\
\end{itemize}

Si les deux droites ne sont pas parallèles, il faut remplacer une variable de $D_1$ dans $D_2$
ou inversément pour trouver la deuxième variable. On a choisit de remplacer $y$.
On a donc l'égalité suivante : 

$$ x = \frac{(c_1 * b_2 - c_2 * b_1)}{(a_2 * b_1 - a_1 * b_2)} $$

\begin{proof}
$$ D_1 \equiv y = \frac{-a_1 * x}{b_1} - \frac{c_1}{b_1} $$
$$ D_2 \equiv y = \frac{-a_2 * x}{b_2} - \frac{c_2}{b_2} $$
On remplace $y$ de $D_1$ dans $D_2$ :
$$ \frac{-a_1 * x}{b_1} - \frac{c_1}{b_1} = \frac{-a_2 * x}{b_2} - \frac{c_2}{b_2} $$
Il ne reste plus qu'à simplifier :
$$ \frac{-a_1 * x}{b_1} + \frac{a_2 * x}{b_2} = \frac{c_1}{b_1} - \frac{c_2}{b_2} $$
$$ \frac{(-a_1 * x * b_2) + (a_2 * x * b_1)}{(b_1 * b_2)} = \frac{(c_1 * b_2 - c_2 * b_1)}{(b_1 * b_2)} $$
$$ (a_2 * b_1 * x) - (a_1 * b_2 * x) = (c_1 * b_2 - c_2 * b_1) $$
$$ x * ((a_2 * b_1) - (a_1 * b_2)) = (c_1 * b_2 - c_2 * b_1) $$
$$ x = \frac{(c_1 * b_2 - c_2 * b_1)}{(a_2 * b_1 - a_1 * b_2)} $$ \\
\end{proof}

Maintenant qu'on a la coordonnée $x$, il suffit de la remplacer
dans une des deux équations $D_1$ ou $D_2$ pour trouver le $y$ 
correspondant.

\subsubsection{Trouver l'intersection entre une droite et un segment de droite}

L'intersection entre une droite et un segment ressemble très fort
à l'intersection entre deux droites tout simplement car il faudra transformer
le segment en droite pour trouver une intersection entre ces deux droites
puis ensuite de vérifier si cette intersection se trouve bien dans le domaine
$x$ et $y$ du segment.

La différence est que l'intersection peut être un segment entier si la droite
et le segment sont confondu.

Pour transformer un segment en droite : \\
Soit $A$ et $B$ les deux extrémités du segments, les paramètres de la droite valent : 
$ a = A.y - B.y $ \\
$ b = B.x - A.x $ \\
$ c = (A.x * B.y) - (A.y * B.x) $ \\

\subsubsection{Trouver l'intersection entre deux segments}

Dans le cas de l'intersection segment/segment, il faut transformer les
deux segments en droites, s'il y a intersections : vérifier que le point
d'intersection appartient aux domaines $x$ et $y$ des deux segments. \\
De plus, si les deux droites sont confondues, l'intersection sera un segment.

\subsubsection{Trouver le(s) intersection(s) entre une ellipse et une droite}

La formule d'une ellipse est : 

$$ E \equiv \frac{(x - x1)^2}{a^2} + \frac{(y - y1)^2}{b^2} = 1 $$

\begin{description}
\item où $x1$ et $y1$ sont respectivement les coordonnées
$x$ et $y$ du centre de l'ellipse. 
\item où $a$ et $b$ sont respectivement les rayons de l'axe
$x$ et $y$. \\
\end{description}

Pour trouver une intersection entre une ellipse
et une droite, il faut égaler deux variables 
identiques :

On doit donc remplacer la variable $x$ ou $y$ de la droite dans
l'équation de l'ellipse.

Le cas de la droite verticale : 

Dans ce cas-là, il n'y a pas de choix, il faut remplacer
$x$ dans l'équation de l'ellipse.
Nous avons également décidé de refactoriser l'équation en 
prenant le PPCM (Plus Petit Commun Multiple) de $a^2 * b^2$ que
nous appellerons ici $lcm$ pour Least Commun Multiple.
Ce choix a été fait pour éviter les overflows lorsque 
de nombres trop grands sont mis au carré et multipliés.
Bien que dans notre cas, nous avons rarement des nombres
pouvant fournir de tels résultats.

$$ E \equiv (lcmy \cdot (k - x1)^2) + ((y - y1)^2 \cdot lcmx) = lcm $$  
où $lcmy$ est le facteur par lequel il faut multiplier
$b^2$ (rayon $y$ au carré) pour obtenir $lcm$, \\
où $lcmx$ est le facteur par lequel il faut multiplier
$a^2$ (rayon $x$ au carré) pour obtenir $lcm$.

$$ E \equiv lcmy \cdot (k-x1)^2 + (y^2 + y1^2 - 2 \cdot y1 \cdot y)
\cdot lcmx = lcm$$
$$ E \equiv lcmy \cdot (k-x1)^2 + lcmx \cdot y^2 + lcmx \cdot
y1^2 - 2 \cdot lcmx \cdot y1 \cdot y - lcm = 0 $$

Avec ceci, il reste plus qu'à résoudre l'équation du second degré 
avec 
$$ \rho = b^2 - 4ac$$
où $$ a = lcmx $$
$$ b = 2 \cdot lcmx \cdot y1 \cdot y $$
$$ c = (lcmy \cdot (k-x1)^2) + (lcmx \cdot y1^2) - lcm $$



Le nombre d'intersections est différent selon la valeur de $\rho$.

\[
	n =
	\begin{cases}
		0 & \text{si } \rho < 0 \\
		1 & \text{si } \rho = 0 \\
		2 & \text{si } \rho > 0  
	\end{cases}
\]

$$ y = \frac {-b}{2a} $$

$$ y1 = \frac{(-b + \sqrt{\rho})}{2a} $$
$$ y2 = \frac{(-b - \sqrt{\rho})}{2a} $$

On a donc le(s) $y$ du/des point(s) d'intersection, et le $x$
vaut $k$ (de l'équation de départ).



Le cas de la droite non verticale : 


C'est le même principe que le cas de la droite verticale sauf
que pour trouver la deuxième variable finale il faudra
remplacer la variable trouvée dans l'équation de la droite.

\subsubsection{Trouver le(s) intersections entre une ellipse et un segment de droite}

Il s'agit du même principe que les intersections droite/segment.
Il faut vérifier les intersections entre les ellipses et un segment
transformé en droite et ensuite vérifier si les points d'intersections
se trouvent dans le domaine du segment.

\subsubsection{Trouver le(s) intersection(s) entre un rectangle et une droite}

\subsubsection{Trouver le(s) intersection(s) entre un rectangle et un esgment de droite}



\end{document}
