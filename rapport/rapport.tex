\documentclass[]{article}
\usepackage{amssymb,amsmath}

\usepackage[utf8]{inputenc}
\usepackage[francais]{babel}

\usepackage[unicode=true,
	colorlinks=true,
linkcolor=blue]{hyperref}

\hypersetup{breaklinks=true, pdfborder={0 0 0}}
\setlength{\parindent}{0pt}
\setlength{\parskip}{6pt plus 2pt minus 1pt}
\setlength{\emergencystretch}{3em}  % prevent overfull lines
\setcounter{secnumdepth}{0}

\title{Rapport du projet Starlight}
\author{Florian Knop (39310) - Gatien Bovyn (39189)}

\begin{document}
\maketitle

\newpage

\tableofcontents

\newpage

\section{Introduction}


Ce document vise à présenter le travail d’analyse et de programmation effectué
lors de la réalisation du projet du laboratoire Langage C++ : Starlight.

Ce projet a été réalisé en binôme par Florian Knop, matricule 39310 groupe 2G13,
et Gatien Bovyn, matricule 39189 groupe 2G11.

Le programme à concevoir consiste en une implémentation du modèle et d’une interface
graphique du jeu baptisé Starlight, puzzle à 2 dimensions basé sur la lumière.


\section{Sections}

\section{Conventions de nommages utilisées}

Dans cette section, nous présenterons les différentes conventions 
utilisées lors de ce projet. Nous avons décidé de reprendre certaines
conventions utilisées par la STL tout en gardant d'autres conventions.

De manière globale, tous les noms de variables, classes, fichiers, etc.
sont en anglais.

\subsection{Nommage des fichiers}

Les noms de fichiers sont entièrement en minuscules
et possèdent les extensions .h
pour les headers et .cpp pour les fichiers sources.

\begin{itemize}
	\item \textbf{nomfichier.h}
	\item \textbf{nomfichier.cpp}
\end{itemize}

\subsection{Classes}

Les noms des classes commencent par une majuscule à chaque mot.
Cette convention est également valable pour les noms 
d'énumérations et structures.

\begin{itemize}
	\item \textbf{Classe}
	\item \textbf{NomClasse}
	\item \textbf{NomStruct}
	\item \textbf{NomEnumeration}
\end{itemize}

\newpage

\subsection{Variables}

\subsubsection{Variables de classe}

Les noms des variables de classes sont en minuscules, les mots sont 
séparés par des underscores et le nom finit par un underscore.

\begin{itemize}
	\item \textbf{variable\_}
	\item \textbf{nom\_variable\_}
\end{itemize}

Cette convention est également valable pour les variables d'une
structure.

\subsubsection{Variables locales}

Les noms des variables locales commencent par une minuscule, et par
la suite d'une majuscule à chaque début de mot.

\begin{itemize}
	\item \textbf{variable}
	\item \textbf{nomVariable}
\end{itemize}

\subsubsection{Constantes}

Les noms des constantes sont entièrement en majuscules et les mots sont
séparés par des underscores.

\begin{itemize}
	\item \textbf{CONSTANTE}
	\item \textbf{NOM\_CONSTANTE}
\end{itemize}

\subsection{Méthodes}

D'une manière générale, les noms de méthodes sont en minuscules et
séparés par des underscores.

\subsubsection{Getters}

Le nom d'un getter (accesseur en lecture de variable de classe/structure)
est égal au nom 
de la variable de classe sans l'underscore final.

\begin{itemize}
	\item Variable : \textbf{variable\_} - Getter : \textbf{variable()}
	\item Variable : \textbf{nom\_var\_} - Getter : \textbf{nom\_var()}
\end{itemize}

\subsubsection{Setters}

Le nom d'un setter (accesseur en écriture de variable de classe/structure)
est égal au nom de la variable sans l'underscore final précédé
d'un set.

\begin{itemize}
	\item Variable : \textbf{variable\_} - Setter : \textbf{set\_variable()}
	\item Variable : \textbf{nom\_var\_} - Setter : \textbf{set\_nom\_var()}
\end{itemize}

\subsubsection{Autres méthodes}

Les autres méthodes possèdent les mêmes conventions que celles
énoncées ci-dessus.

\begin{itemize}
	\item \textbf{methode()}
	\item \textbf{nom\_methode()}
\end{itemize}

\subsection{Éléments d'une énumération} 

Les éléments d'une énumération suivent les mêmes conventions que les 
conventions des constantes énoncées ci-dessus.

\section{Présentation des différentes classes}

Dans cettte section, nous allons décrire les différentes classes
composants ce projet.
L’implémentation du projet est divisée entre la partie modèle et la partie vue.
Elle est également basée sur le design pattern  Observeur / Observé  comme demandé
dans les consignes.

\subsection{Modèle}

Dans cette section, nous allons décrire les différentes classes du modèle 
(classes métiers).
Un squelette de classes a été fourni par Monsieur Absil.
Ce squelette contenait les fichiers suivants :
`point.h, source.h, dest.h, nuke.h, wall.h,
crystal.h, lens.h, mirror.h, ray.h, level.h`.
Nous avons décidé de modifier ce squelette tout en gardant la 
structure générale.

\subsubsection{Line}


Cette classe représente une droite, elle possède un point et un
angle. Gràce à cela on peut trouver n'importe quel point de la 
droite gràce en ayant la distance entre le point d'origine
et le point d'arrivée.

Dans les méthodes intersects(...) de la classe Line, un passage
par pointeur de pointeur est fait car un pointeur est passé par 
valeur et lors de l'initialisation il ne pointera donc pas la même
adresse mémoire que le pointeur d'origine.
Si on passait donc par simple pointeur, notre pointeur copié
aurait une bonne zone mémoire mais notre pointeur d'origine
resterait à *nullptr*.

\subsubsection{LineSegment}

Cette classe représente un segment de droite, elle possède
deux points qui représentent les extrémités du segment.

\subsubsection{Ellipse}


Cette classe représente une conique de forme elliptique.
C'est à dire une ellipse ou un cercle.

\subsubsection{Rectangle}

Cette classe représente une forme géométrique rectangulaire.
Elle possède un point supérieur gauche ainsi qu'une longueur et hauteur 
nous permettant de retrouver facilement les autrex extrémités de la 
forme.

Cette classe sert à représenter la classe Dest (la destination) qui est
le seul objet de forme rectangulaire.
	
\subsection{Vue}


L’interface graphique a été réalisée en `Qt` à la main.

Les classes composant la partie vue de l’application sont :

\subsubsection{DestinationView}

\paragraph{Description}


Classe modélisant la destination à atteindre par le rayon émis depuis la source pour gagner la partie.

\paragraph{MapView}

\paragraph{MirrorView}

\paragraph{NukeView}

\paragraph{SourceView}

\paragraph{WallView}

\section{Conclusion}

\section{Bibliographie}

\section{Annexes}

\subsection{Démarches mathématique}

\subsubsection{Trouver l'intersection entre deux droites}



\begin{equation} \label{eq:droite}
	D \equiv y = ax + b 
\end{equation}
\begin{equation} \label{eq:droiteverticale}
	D \equiv x = k 
\end{equation}


\eqref {eq:droite} 
représente une droite non verticale.
\eqref {eq:droiteverticale} 
représente une droite verticale où k est une valeur
quelquonque sur l'axe des x.

Dans un premier temps, il faut tester que la pente $a$ est différente. 
Si cette dernière est égale, il n'y a pas d'intersections. En effet,
si les pentes sont les mêmes, cela signifie que les droites sont soit
parallèles soit égales. Dans le cas de droites égales, on considère qu'il
n'y a pas d'intersections, bien qu'en réalité il y en a une infinité.

Si la pente est différente,
l'intersection entre deux droites est assez simple, il suffit 
d'injecter une variable d'une des deux équations dans l'autre.
Dans le cadre du projet, il faut evidemment faire attention aux 
droites verticales.

Le cas d'une droite verticale : 

On injecte \eqref{eq:droiteverticale} dans \eqref{eq:droite}.

\begin{equation} \label{eq:interv}
	D \equiv y = ak + b
\end{equation}

Avec \eqref{eq:interv} on obtient $y$ du point dont le $x = k$
pour une droite verticale \eqref{eq:droiteverticale}.

Pour deux droites non verticales on injecte le $y$ d'une équation
de type \eqref{eq:droite} dans le $y$ d'une autre équation de typ \eqref{eq:droite}.

\begin{equation} \label{eq:internv}
	a remplir
\end{equation}

$ x = \frac {(y - b)}{ a} $ pour une droite non verticale
dont le y est celui obtenu plus tôt \eqref{eq:droite}.

\subsubsection{Trouver l'intersection entre une droite et un segment}

\subsubsection{Trouver le(s) intersections entre une ellipse et une droite}

La formule d'une ellipse est : 

$$ E \equiv \frac{(x - x1)^2}{a^2} + \frac{(y - y1)^2}{b^2} = 1 $$

\begin{description}
\item où $x1$ et $y1$ sont respectivement les coordonnées
$x$ et $y$ du centre de l'ellipse. \\
\item où $a$ et $b$ sont respectement les rayons de l'axe
$x$ et $y$. \\
\end{description}

Pour trouver une intersection entre une ellipse
et une droite, il faut égaler deux variables 
identiques :


On doit donc remplacer la variable $x$ ou $y$ de la droite dans
l'équation de l'ellipse.

Le cas de la droite verticale : 

Dans ce cas la, il n'y a pas de choix, il faut remplacer
$x$ dans l'équation de l'ellipse.
Nous avons également décidé de refactoriser l'équation en 
prenant le PPCM (Plus petit commun multiple) de $a^2 * b^2$ que
nous apellerons ici $lcm$ pour Least Commun Multiple.
Ce choix a été fait pour éviter les overflows lorsque 
de nombres trop grands sont mis au carré et multipliés.
Bien que dans notre cas, on nous avons rarement des nombre
pouvant obtenir un tel résultat.

$$ E \equiv (lcmy \cdot (k - x1)^2) + ((y - y1)^2 \cdot lcmx) = lcm $$  
où $lcmy$ est le facteur par lequel il faut multiplier
$b^2$ (rayon $y$ au carré) pour obtenir $lcm$, \\
où $lcmx$ est le facteur par lequel il faut multiplier
$a^2$ (rayon $x$ au carré) pour obtenir $lcm$.

$$ E \equiv lcmy \cdot (k-x1)^2 + (y^2 + y1^2 - 2 \cdot y1 \cdot y)
\cdot lcmx = lcm$$
$$ E \equiv lcmy \cdot (k-x1)^2 + lcmx \cdot y^2 + lcmx \cdot
y1^2 - 2 \cdot lcmx \cdot y1 \cdot y - lcm = 0 $$

Avec ceci, il reste plus qu'à résoudre l'équation du second degré 
avec 
$$ \rho = b^2 - 4ac$$
où $$ a = lcmx $$
$$ b = 2 \cdot lcmx \cdot y1 \cdot y $$
$$ c = (lcmy \cdot (k-x1)^2) + (lcmx \cdot y1^2) - lcm $$



Le nombre d'intersections est différent selon la valeur de $\rho$.

\[
	n =
	\begin{cases}
		0 & \text{si } \rho < 0 \\
		1 & \text{si } \rho = 0 \\
		2 & \text{si } \rho > 0  
	\end{cases}
\]

$$ y = \frac {-b}{2a} $$

$$ y1 = \frac{(-b + \sqrt{\rho})}{2a} $$
$$ y2 = \frac{(-b - \sqrt{\rho})}{2a} $$

On a donc le(s) $y$ du/des points d'intersections, et le $x$
vaut $k$ (de l'équation de départ).



Le cas de la droite non verticale : 


C'est le même principe que le cas de la droite verticale sauf
que pour trouver la deuxième variable finale il faudra
remplacer la variable trouvée dans l'équation de la droite.

\subsubsection{Trouver le(s) intersections entre une ellipse et un segment}

Il s'agit du même principe que les intersections droite/segment.
Il faut vérifier les intersections entre les ellipses et un segment
transformé en droite et ensuite vérifier si les points d'intersections
se trouvent sur le segment.

\end{document}
