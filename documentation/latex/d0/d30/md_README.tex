
\begin{DoxyItemize}
\item Knop Florian
\item Bovyn Gatien
\end{DoxyItemize}

\section*{T\+O\+D\+O}

{\bfseries Une fois effectué -\/$>$ Barrer la ligne !}


\begin{DoxyItemize}
\item Editeur \+: Prévenir les erreurs (pour éviter de sauvegarder un objet foireux)
\item $\sim$$\sim$\+Bloquer le mapview une fois la partie gagnée ou perdue.$\sim$$\sim$
\item $\sim$$\sim$\+Editeur \+: pouvoir bouger les autres éléments au clavier.$\sim$$\sim$
\item $\sim$$\sim$\+Editeur \+: Modifier dynamiquement les propriétés ?$\sim$$\sim$
\item $\sim$$\sim$\+Editeur \+: Message de confirmation quand on quitte ? Pour éviter de perdre des maps non sauvegardée$\sim$$\sim$
\item $\sim$$\sim$\+Editeur \+: Ctrl-\/\+X pour delete un objet ? \+:D$\sim$$\sim$
\item $\sim$$\sim$\+Editeur \+: Simple clic pour le sélectionner (c'est un peu galère) -\/$>$ O\+S\+E\+F$\sim$$\sim$
\item $\sim$$\sim$\+Gérer les collisions entre miroirs de même angle.$\sim$$\sim$
\item $\sim$$\sim$\+Gérer la reflexion du rayon sur un miroir. A F\+I\+N\+I\+R$\sim$$\sim$
\item $\sim$$\sim$\+Gérer les divers évènements lors de l'intersection avec un objet.$\sim$$\sim$
\item $\sim$$\sim$\+Faire le cast d'un élément en sous-\/élément$\sim$$\sim$ !
\item $\sim$$\sim$\+Gérer les lignes verticales.$\sim$$\sim$
\item $\sim$$\sim$\+Gérer les intersections entres droites, segments, ellipses.$\sim$$\sim$
\item $\sim$$\sim$\+Gérer le retour du point d'intersection et de l'objet.$\sim$$\sim$
\item $\sim$$\sim$\+Gérer les vérifications dans les constructeurs. -\/ A F\+I\+N\+I\+R$\sim$$\sim$
\item $\sim$$\sim$\+Gérer le déplacement des miroirs.$\sim$$\sim$
\item $\sim$$\sim$\+Gérer la rotations des miroirs. -\/ A T\+E\+S\+T\+E\+R$\sim$$\sim$
\end{DoxyItemize}

\section*{Rules}

Starlight est un puzzle à deux dimensions se jouant sur une carte rectangulaire. Le but du jeu est de dévier un rayon lumineux d'une source vers une cible en évitant certains obstacles. Plus particulièrement, on trouve les éléments suivants sur une carte.


\begin{DoxyItemize}
\item Une unique source \+: cet élément émet un rayon lumineux d'une longueur d'onde donnée sous un certain angle.
\item Une unique cible (ou destination) \+: cet élément doit être éclairé par un rayon lumineux pour remporter la partie.
\item Un ensemble de miroirs \+: un miroir est un objet réfléchissant la lumière d'un seul côté suivant le schéma naturel de la réflexion de la lumière. Plus particulièrement, un rayon incident à un miroir sous un angle \&\#x3b8;i sera réfléchi sous le même angle \&\#x3b8;r.
\item Un ensemble de murs \+: les murs ne réfléchissent pas la lumière. Tout rayon incident à un mur ne se propage pas, et s'arrête donc là où il y est incident.
\item Un ensemble de lentilles. Les lentilles sont des objets transparents qui ne laissent passer un rayon lumineux que dans un certain intervalle de longueur d'onde \mbox{[}m,n\mbox{]}. Si un rayon lumineux possède une longueur d'onde \&\#x3d1; telle que m \&\#x2264; \&\#x3d1; \&\#x2264; n, il traverse la lentille sans subir aucune modification. Sinon, la lentille stoppe le rayon (elle se comporte comme un mur).
\item Un ensemble de cristaux \+: un cristal est un élément transparent qui modifie la longueur d'onde d'un rayon, en l'augmentant ou la diminuant. Tout rayon qui traverse un cristal le traverse donc sans subir de modification de trajectoire, mais voit sa longueur d'onde modifiée.
\item Un ensemble de bombes. Les bombes sont des objets qui, si éclairés, explosent et font automatiquement perdre la partie au joueur.
\item Un ensemble de rayons. Initialement émis par la source du jeu, ils sont rectilignes et se réfléchissent sur les miroirs. Un rayon est donc un segment de droite. Un rayon possède également une autre caractéristique \+: sa longueur d'onde. La longueur d'onde d'un rayon permet de déterminer, comme mentionné ci-\/dessus, si oui ou non un rayon traverse une lentille. Elle est modifiée par un cristal.
\end{DoxyItemize}

Tous les objets ci-\/dessus sont immobiles, à l'exception des miroirs qui peuvent être déplacés et pivotés dans certaines limites. Les rayons lumineux ne sont présents dans le jeu que lorsque la source lumineuse est allumée.





\section*{Exigences et remise}


\begin{DoxyItemize}
\item Fournir une interface graphique permettant de jouer à Starlight en respectant les règles décrites dans ce document et les propriétés de chacun des éléments de la carte.
\item L'interface graphique doit être batie sur l'architecture {\itshape Observateur/observé}
\item Être capable de lire et de charger des cartes depuis un fichier texte dont le format est décrit.
\item Le joueur doit être capable d'allumer et d'éteindre la source. Les rayons sont automatiquement calculés quand la source est allumée. Ils disparaissent quand la source est éteinte.
\item {\bfseries Remise le 24 avril 2015}
\end{DoxyItemize}





\subsection*{Points facultatifs}

\subsubsection*{Objectifs techniques}


\begin{DoxyItemize}
\item Changer la couleur du rayon en fonction de sa longueur d'onde.
\item Changer l'apparence des rayons lumineux et des éléments de la carte, par exemple via des textures (images).
\item Ajouter des effets sonores.
\item {\bfseries Fournir un éditeur de carte intuitif, par exemple, permettant de placer les composants à la souris. Sa modélisation et conception sont laissées à votre discrétion.}
\end{DoxyItemize}

\subsubsection*{Objectifs algorithmiques}


\begin{DoxyItemize}
\item Permettre de changer les formes de base des éléments du jeu.
\item Gérer les collisions entre les miroirs et les autres éléments du jeu. Par exemple, empêcher que l'extrémité des miroirs ne rentre dans les murs.
\item {\bfseries Fournir un générateur de carte aléatoire simple. Remarque\+: ce point est relativement difficile. N'essayez que si vous avez fini les objectifs secondaires précédents.}
\item {\bfseries Implémenter un algorithme permettant de résoudre le puzzle automatiquement. Remarque \+: ce point est très difficile. N'essayez que si vous n'avez rien d'autre à faire.}
\end{DoxyItemize}

\subsubsection*{Compilation}

{\itshape Nécessite g++-\/4.9}

Installation sous Ubuntu (testé 12.\+04) \+:

``` sudo add-\/apt-\/repository ppa\+:ubuntu-\/toolchain-\/r/test sudo apt-\/get update sudo apt-\/get install g++-\/4.9 cd /usr/bin sudo rm g++ sudo ln -\/s g++-\/4.9 g++ ```

\subsubsection*{Liens utiles}


\begin{DoxyItemize}
\item \href{http://i.imgur.com/Vl122xn.png}{\tt http\+://i.\+imgur.\+com/\+Vl122xn.\+png}
\item Solution \hyperlink{classLevel}{Level} 1 \+: \href{http://i.imgur.com/StfYIJY.png}{\tt http\+://i.\+imgur.\+com/\+Stf\+Y\+I\+J\+Y.\+png} 
\end{DoxyItemize}